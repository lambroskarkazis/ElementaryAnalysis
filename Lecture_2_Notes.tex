\documentclass{article}
\usepackage{amsmath,amsthm,amssymb}

\title{MATH 325 - Lecture 2}
\author{Lambros Karkazis}
\date{September 23, 2017}

\begin{document}

\maketitle
\section{Logical Operators}
\underline{Negation:} NOT, \textasciitilde, $\neg$\\
\underline{Conjunction:} AND, $\wedge$\\
\underline{Disjunction:} OR, $\vee$\\
\underline{Implication:} IF... THEN..., $\implies$\\
\underline{Equivalence:} ...IF AND ONLY IF..., IFF $\iff$\\
\underline{Precedence:} ($1^{st}$) NOT, AND, OR, IF... THEN..., IFF (Last)\\

\section{Example 4}
\subsection{Logical Analog of DeMorgan's Laws (p.33)}
Logical Definition of DeMorgan's Laws
\begin{align*}
	&\neg(P \vee Q) \iff \neg P \wedge \neg Q\\
	&\neg(P \wedge Q) \iff \neg P \vee \neg Q\\
\end{align*}
Truth Table proving $\neg(P \wedge Q) \iff \neg P \vee \neg Q$
\begin{center}
        \begin{tabular}{||c | c | c | c | c | c | c ||} 
        \hline\hline
        $P$ & $Q$ & $\neg P$ & $\neg Q$ & $P \wedge Q$ &$ \neg(P \wedge Q)$ & $\neg P \vee \neg Q$ \\ [0.5ex] 
        \hline\hline
        F & F & T & T & F & \textbf{T} & \textbf{T}\\ 
        \hline
        F & T & T & F & F & \textbf{T} & \textbf{T}\\
        \hline
        T & F & F & T & F &\textbf{T} & \textbf{T}\\
        \hline
        T & T & F & F & T & \textbf{F} & \textbf{F}\\
        \hline\hline
        \end{tabular}
\end{center}

\subsection{The Negation of $P \implies Q$}
$P \wedge \neg Q \iff \neg(P \implies Q)$

\begin{center}
	\begin{tabular}{|| c | c | c | c | c | c ||}
	\hline\hline
	$P$ & $Q$ & $\neg Q$ & $P \wedge \neg Q$ & $P \implies Q$ & $\neg (P \implies Q)$ \\ [0.5ex]
	\hline\hline
	T & T & F & \textbf{F} & T & \textbf{F}\\
	T & F & T & \textbf{T} & F & \textbf{T}\\
	F & T & F & \textbf{F} & F & \textbf{F}\\
	F & F & T & \textbf{F} & T & \textbf{F}\\
	\hline\hline
	\end{tabular}
\end{center}

Note:
\begin{align*}
	& \neg(P \implies Q) \iff P \wedge \neg Q && \text{Negation of an Implication}\\
	& \neg[\neg(P \implies Q)] \iff \neg(P \wedge \neg Q) && \text{Double Negation}\\
	& P \implies Q \iff \neg P \vee Q && \text{Equivalent Form of Implication}\\
\end{align*}

\section{Definition 5 (p. 34):}
Suppose $P$ and $Q$ are statements.
\begin{enumerate}
	\item Contrapositive of $P \implies Q$ is $\neg Q \implies \neg P$
	\item Converse of $P \implies Q$ is $Q \implies P$
	\item Inverse of $P \implies Q$ is $\neg P \implies \neg Q$
	\item Negation of $P \implies Q$ is $P \wedge \neg Q$
\end{enumerate}

\subsection{Remark 6:}
\begin{itemize}
	\item The contrapositive is always equivalent ot the original implication.
	\item The converse and inverse are not equivalent to an implication.
	\item The converse and inverse are contrapositives.
	\item To prove equivalence, once ust prove implication and its converse.
\end{itemize}

\section{Variables and Quanitifiers (sec 1.4)}
Make statements that depend on an unspecified parameter, a \underline{variable}, but a context is needed to determine the truth value.
\subsection{Example 7}
$P(x) = x^2 - 5x + 6 = 0$\\
$P(x)$ is a sentence, but not a statement. You need information about x to determine its truth value.\\
Need to know what ``universe'' x lives in.\\
$P(0), P(2)$ are statements:
$P(0) = 0^2 - 5*(0) + 6 = 0$\\
$6 \not= 0$\\
False\\\\
$P(2) = 2^2 - 2*(5) + 6 = 0$\\
$0 = 0$
True\\\\
Quantifiers are used to provide a larger context for variables\\
\subsection{Definition 8 (p 33)}
\begin{itemize}
	\item Phrases such as ``For all...'', ``For every...'' are \underline{universal quantifiers}: $\forall$
	\item Phrases such as ``There exists...'', ``There is at least one...'' are \underline{existential quantifiers}: $\exists$
\end{itemize}
\subsubsection{More Notation:}
$\ni$, s.t.: ``such that''\\
$\exists !$: ``there exists a unique (exactly one)...''\\\\

\textbf{WARNING:} Unless otherwise stated, all variables are real numbers.
\subsubsection{Convention:}
If a variable appears in an antecedent of an implication without a quantifier, then we assume there is a universal quantifier. $x > 1 \implies x^2 > 1$ really means $\forall x \in \mathbb{R}, x > 1 \implies x^2 > 1$
\subsubsection{Example:}
$P(x) = x^2 - 5x + 6 = 0$\\
(Assuming x is a real number)\\
$\forall x \in \mathbb{R}, P(x) = x^2 - 5x + 6 = 0$ means...\\
``For all real numbers $x$, $x^2 - 5x + 6 = 0$''\\
(False statement because some real numbers $x$ don't satisfy $P(x)$)\\\\
$\exists x \ni P(x)$ means...\\
``There is a real number x s.t. $x^2 - 5x + 6 = 0$\\
(True statement since $P(2)$ evalulates to true)\\\\
$\exists ! x \ni P(x) = 0$ means...\\
``There exactly one real s.t. $x^2-5x+6=0'$''\\
(False statement since $P(2)$ and $P(3)$ both evaluate to true)\\\\
\textbf{WARNING:} $\forall x, \exists y$ is not the same as $\exists y, \forall x$
\subsubsection{Example 8: (Prob 4)}
Suppose that $P(x, y)$ is a statement for each x and y.\\
$\forall x, y, P(x, y) \iff \forall x, \forall y P(x, y) \iff \forall y, \forall x, P(x, y)$\\
Means $P(x, y)$ is true regardless of what x and y are (order does not matter for the same quantifier).\\
$\exists x,y \ni P(x,y) \iff \exists x \ni \exists y \ni P(x, y) \iff \exists y \ni \exists x \ni P(x, y)$\\
Means there's atleast one x and one y that satisfies $P(x, y)$ (order does not matter for the same quantifier).\\
$\forall x, \exists y \ni P(x,y) \not\iff \exists \ni \forall x, P(x,y)$\\
$\forall x, \exists y \ni P(x, y)$ $y$ can depend on $x$.\\
$\exists y \ni \forall x, P(x, y)$ $y$ cannot depend on x. There is some $y$ s.t. no matter what $x$, $P(x, y)$ holds.
\end{document}